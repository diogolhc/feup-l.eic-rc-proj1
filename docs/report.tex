\documentclass[a4paper,11pt,portuguese]{article}
\usepackage{hyperref}
\usepackage{listings}
\usepackage{xcolor}
\usepackage[portuguese]{babel}

%%% Estilo de código %%%

\definecolor{codegreen}{rgb}{0,0.6,0}
\definecolor{codegray}{rgb}{0.5,0.5,0.5}
\definecolor{codepurple}{rgb}{0.58,0,0.82}
\definecolor{backcolour}{rgb}{0.95,0.95,0.92}

\lstdefinestyle{mystyle}{
    backgroundcolor=\color{backcolour},   
    commentstyle=\color{codegreen},
    keywordstyle=\color{magenta},
    numberstyle=\tiny\color{codegray},
    stringstyle=\color{codepurple},
    basicstyle=\ttfamily\footnotesize,
    breakatwhitespace=false,         
    breaklines=true,                 
    captionpos=b,                    
    keepspaces=true,                 
    numbers=left,                    
    numbersep=5pt,                  
    showspaces=false,                
    showstringspaces=false,
    showtabs=false,                  
    tabsize=2
}

\lstset{style=mystyle}


\begin{document}

%%% Identificação %%%

\author{
    Diogo Costa\\
    \href{mailto:up201906731@edu.fe.up.pt}{up201906731@edu.fe.up.pt}
    \and
    Francisco Colino\\
    \href{mailto:up201905405@edu.fe.up.pt}{up201905405@edu.fe.up.pt}
}
\title{FEUP -- Redes de Computadores \large 2021/2022 \\ \large 1.º Trabalho Laboratorial}
\date{\today}
\maketitle

%%% Sumário %%%
\begin{center}
    \textbf{Sumário}
\end{center}

%TODO


%%% Introdução %%%
\section{Introdução}

    O objetivo deste trabalho é implementar um protocolo de ligação de dados, 
    de acordo com o guião fornecido, que permite fazer a transmissão de ficheiros 
    de forma assíncrona através de portas série assegurando a integridade dos ficheiros.
    Esta integridade deve ser assegurada mesmo com interrupções e interferências. 
    Este relatório procura expor a teoria por de trás deste projeto, como é que os
    objetivos foram alcançados e os testes efetuados à eficiência do protocolo. \par

    Este relatório está estruturado da seguinte forma:
    \begin{itemize}
        \item \textbf{Arquitetura} -- Blocos funcionais e interfaces.
        
        \item \textbf{Estrutura do Código} -- Demonstração das APIs, principais 
        estruturas de dados, principais funções e a sua relação com a arquitetura.

        \item \textbf{Casos de uso principais} -- Identificação dos casos de uso 
        e representação das sequências de chamada de funções.

        \item \textbf{Protocolo de ligação lógica} -- Identificação dos principais
        aspetos funcionais da ligação lógica e descrição das estratégias usadas na
        implementação destes aspetos com extratos de código.

        \item \textbf{Protocolo de aplicação} -- Identificação dos principais
        aspetos funcionais da aplicação e descrição das estratégias usadas na
        implementação destes aspetos com extratos de código.

        \item \textbf{Validação} -- Descrição dos testes efetuados com apresentação
        quantificada dos resultados.

        \item \textbf{Eficiência do protocolo de dados} -- Caraterização estatística 
        da eficiência do protocolo, efetuada recorrendo a medidas sobre o código desenvolvido.

        \item \textbf{Conclusão} -- Síntese da informação apresentada nas secções
        anteriores e reflexão sobre os objetivos de aprendizagem alcançados.
        
    \end{itemize}


%%% Arquitetura %%%
\section{Arquitetura}

%TODO


%%% Estrutura do código %%%
\section{Estrutura do código}

%TODO


%%% Casos de uso principais %%%
\section{Casos de uso principais}

%TODO


%%% Protocolo de ligação lógica %%%
\section{Protocolo de ligação lógica}

%TODO


%%% Protocolo de aplicação %%%
\section{Protocolo de aplicação}

%TODO


%%% Validação %%%
\section{Validação}

%TODO


%%% Eficiência do protocolo de ligação de dados %%%
\section{Eficiência do protocolo de ligação de dados}

%TODO


%%% Conclusões %%%
\section{Conclusões}

%TODO


%%% Anexos %%%
\newpage

\section{Anexos}
\subsection{Código Fonte}

\lstinputlisting[language=C, caption=sender.c]{../src/sender.c}

\lstinputlisting[language=C, caption=receiver.c]{../src/receiver.c}

\lstinputlisting[language=C, caption=aplic.h]{../src/aplic.h}

\lstinputlisting[language=C, caption=linklayer.h]{../src/linklayer.h}

\lstinputlisting[language=C, caption=aplic.c]{../src/aplic.c}

\lstinputlisting[language=C, caption=linklayer.c]{../src/linklayer.c}


\end{document}
