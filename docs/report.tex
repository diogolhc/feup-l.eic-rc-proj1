\documentclass[a4paper,11pt,portuguese]{article}
\usepackage{hyperref}
\usepackage{listings}
\usepackage{xcolor}
\usepackage[portuguese]{babel}

%%% Estilo de código %%%

\definecolor{codegreen}{rgb}{0,0.6,0}
\definecolor{codegray}{rgb}{0.5,0.5,0.5}
\definecolor{codepurple}{rgb}{0.58,0,0.82}
\definecolor{backcolour}{rgb}{0.95,0.95,0.92}

\lstdefinestyle{mystyle}{
    backgroundcolor=\color{backcolour},   
    commentstyle=\color{codegreen},
    keywordstyle=\color{magenta},
    numberstyle=\tiny\color{codegray},
    stringstyle=\color{codepurple},
    basicstyle=\ttfamily\footnotesize,
    breakatwhitespace=false,         
    breaklines=true,                 
    captionpos=b,                    
    keepspaces=true,                 
    numbers=left,                    
    numbersep=5pt,                  
    showspaces=false,                
    showstringspaces=false,
    showtabs=false,                  
    tabsize=2
}

\lstset{style=mystyle}


\begin{document}

%%% Identificação %%%

\author{
    Diogo Costa\\
    \href{mailto:up201906731@edu.fe.up.pt}{up201906731@edu.fe.up.pt}
    \and
    Francisco Colino\\
    \href{mailto:up201905405@edu.fe.up.pt}{up201905405@edu.fe.up.pt}
}
\title{FEUP -- Redes de Computadores \large 2021/2022 \\ \large 1.º Trabalho Laboratorial}
\date{\today}
\maketitle

%%% Sumário %%%
\begin{center}
    \textbf{Sumário}
\end{center}

Este projeto foi realizado como o sendo o 1.º projeto laboratorial da unidade curricular
\textit{Redes de Computadores}, fazendo esta parte da \textit{Licenciatura em Engenharia
Informática e Computação} da \textit{FEUP}. O projeto consistiu na implementação de um
protocolo de ligação de dados que permite a transferência confiável de dados entre dois
computadores através da porta série. Para além deste protocolo foi implementada uma
aplicação de transferência de ficheiros que faz uso do serviço fornecido pelo protocolo
de ligação de dados. \par

Todos os objetivos foram atingidos na medida em que foi implementado com sucesso um
protocolo de ligação de dados confiável e a aplicação que faz uso desse protocolo. Esta
implementação foi testada em contexto laboratorial e provou ser resistente a interrupções
e interferências. Foi ainda feita uma análise estatística experimental e comparados os
resultados aos expectados teoricamente.


%%% Introdução %%%
\section{Introdução}

    O objetivo deste trabalho é implementar um protocolo de ligação de dados, 
    de acordo com o guião fornecido, que permite fazer a transmissão de ficheiros 
    de forma assíncrona através de portas série assegurando a integridade dos ficheiros.
    Esta integridade deve ser assegurada mesmo com interrupções e interferências. 
    Este relatório procura expor a teoria por de trás deste projeto, como é que os
    objetivos foram alcançados e os testes efetuados à eficiência do protocolo. \par

    \hfill \break
    \noindent Este relatório está estruturado da seguinte forma:
    \begin{itemize}
        \item \textbf{Arquitetura} -- Blocos funcionais e interfaces.
        
        \item \textbf{Estrutura do Código} -- Demonstração das \textit{APIs}, principais 
        estruturas de dados, principais funções e a sua relação com a arquitetura.

        \item \textbf{Casos de uso principais} -- Identificação dos casos de uso 
        e representação das sequências de chamada de funções.

        \item \textbf{Protocolo de ligação lógica} -- Identificação dos principais
        aspetos funcionais da ligação lógica e descrição das estratégias usadas na
        implementação destes aspetos com extratos de código.

        \item \textbf{Protocolo de aplicação} -- Identificação dos principais
        aspetos funcionais da aplicação e descrição das estratégias usadas na
        implementação destes aspetos com extratos de código.

        \item \textbf{Validação} -- Descrição dos testes efetuados com apresentação
        quantificada dos resultados.

        \item \textbf{Eficiência do protocolo de dados} -- Caraterização estatística 
        da eficiência do protocolo, efetuada recorrendo a medidas sobre o código
        desenvolvido.

        \item \textbf{Conclusão} -- Síntese da informação apresentada nas secções
        anteriores e reflexão sobre os objetivos de aprendizagem alcançados.
        
    \end{itemize}


%%% Arquitetura %%%
\section{Arquitetura}
    
    O projeto está dividido em dois blocos funcionais principais, \textbf{\textit{Data Link}} e 
    \textbf{\textit{Application}}, podendo desempenhar dois papeis distintos, emissor e recetor.
    Estas duas camadas são independentes com o intuito de tornar o código mais modular de 
    modo a que a camada mais baixo possa ser usado com outras aplicações. \par
    
    A \textbf{camada de ligação de dados \textit{(Data Link)}} é o nível mais baixo. Esta trabalha
    com a porta série e oferece uma interface, que permite a abertura, fecho, leitura e escrita
    numa porta série. Esta permite comunicação assíncrona e fidedigna entre dois
    computadores com a capacidade de deteção e tratamento apropriado de erros, interrupções 
    e interferências sem que haja perdas de dados ou transferência de dados incorretos. \par 

    A \textbf{camada da aplicação \textit{(Application)}} é uma interface que usa a linha de
    comandos para comunicar com o utilizador. Esta oferece dois serviços: emissor e recetor.
    Em ambos, o utilizador tem a liberdade de escolher a porta série a utilizar e, no caso do emissor, 
    escolher o ficheiro a enviar e que nome dar a este no envio ao recetor. A aplicação é também
    responsável pela divisão do ficheiro original em pacotes de um tamanho predefinido para
    envio na camada de ligação de dados. \par 


%%% Estrutura do código %%%
\section{Estrutura do código}

    O código encontra-se dividido em 4 ficheiros \textit{.c} de modo a facilitar a divisão
    nas camadas mencionadas. \par
    
    Ao \textbf{\textit{Data Link}} corresponde o ficheiro \textit{linklayer.c}, à
    \textbf{\textit{Application}} correspondem os ficheiros \textit{aplic.c},
    \textit{receiver.c} e \textit{sender.c}.
    
    \hfill \break
    \noindent Funções principais da camada \textbf{\textit{Data Link}}:
    \begin{itemize}
        \item llopen() -- estabelece a ligação entre as máquinas através de tramas de Supervisão (S)
        \item llwrite() -- envia tramas de Informação (I) e recebe tramas de Supervisão (S)
        \item llread() -- lê tramas de Informação (I) e envia tramas de Supervisão (S)
        \item llclose() -- termina a ligação entre as máquinas através de tramas de Supervisão (S)
    \end{itemize}

    \hfill \break
    \noindent Macros principais da camada \textbf{\textit{Data Link}}:
    \begin{itemize}
        \item BAUDRATE -- valor da \textit{baud rate} a ser utilizada na comunicação
        \item TIME\_OUT\_TIME -- segundos que as funções esperam pelo 
        envio de dados antes de entrarem em \textit{timeout}
        \item MAX\_NO\_TIMEOUT -- número máximo de \textit{timeouts} consecutivos
        \item DATA\_PACKET\_MAX\_SIZE -- tamanho máximo que os dados do campo de informação,
        antes de \textit{stuffing}, podem ter por envio de pacote
    \end{itemize}

    \hfill \break
    \noindent Funções principais da camada \textbf{\textit{Application}}:
    \begin{itemize}
        \item send\_file() -- reparte um ficheiro em pacotes de tamanho predefinido e faz uso da função 
        llwrite() para os enviar
        \item receive\_file() -- recebe diversos pacotes de dados, fazendo uso da função llread(), 
        e organiza-os de forma a montar o ficheiro recebido
    \end{itemize}
    
    \hfill \break
    \noindent Macros principais da camada \textbf{\textit{Application}}:
    \begin{itemize}
        \item CONTROL\_PACKET\_MAX\_SIZE -- tamanho máximo de um pacote de controlo da aplicação
        \item PACKET\_MAX\_SIZE -- tamanho máximo de um pacote da aplicação, obtido como sendo
        o máximo entre o CONTROL\_PACKET\_MAX\_SIZE e o DATA\_PACKET\_MAX\_SIZE
        \item FILE\_NAME\_MAX\_SIZE -- tamanho máximo do nome de um ficheiro em \textit{linux}
    \end{itemize}

%%% Casos de uso principais %%%
\section{Casos de uso principais}

    \subsection{Transmissor}
    
        A aplicação é executada em modo \textbf{\textit{sender}}. O utilizador escolhe a porta série a
        utilizar, o caminho do ficheiro a mandar ao recetor e o nome que deve ser dado ao ficheiro
        na sua receção. Primeiro é estabelecida a ligação entre o transmissor e o recetor, verificando
        que o ficheiro passado à aplicação é válido este é então enviado em pacotes através duma porta
        série fazendo uso do mecanismo \textit{Stop-and-Wait}. Após o envio a ligação é terminada.
        Exemplo: \hfill \break
        ./sender 0 ``pinguim.gif'' ``p1.gif''

        \hfill \break
        \noindent Uma sequência mais detalhada do que acontece:
        \begin{enumerate}
            \item Abre a porta série e estabelece a conexão com fd = llopen(``/dev/ttyS0'', TRANSMITTER)
            \item Envia pacotes de controlo e o ficheiro repartido em pacotes de dados através da função llwrite()
            \item Fecha a porta série, terminando assim a ligação, com llclose()
        \end{enumerate}

    \subsection{Recetor}

        A aplicação é executada em modo \textbf{\textit{receiver}}. O utilizador escolhe a porta série a
        utilizar. Primeiro é estabelecida a ligação entre o transmissor e o recetor e é feita a leitura
        pacote a pacote do ficheiro a ser recebido. Após a leitura a ligação é terminada. Exemplo: \hfill \break
        ./receiver 4

        \hfill \break
        \noindent Uma sequência mais detalhada do que acontece:
        \begin{enumerate}
            \item Abre a porta série e estabelece a conexão com fd = llopen(``/dev/ttyS4'', RECEIVER)
            \item Os pacotes enviados pelo transmissor são lidos sequencialmente através da função llwrite()
            \item Fecha a porta série, terminando assim a ligação, com llclose()
        \end{enumerate}


%%% Protocolo de ligação lógica %%%
\section{Protocolo de ligação lógica}

%TODO


%%% Protocolo de aplicação %%%
\section{Protocolo de aplicação}

%TODO


%%% Validação %%%
\section{Validação}

    De modo a validar o correto funcionamento dos protocolos de ligação de dados
    e de aplicação foram efetuados múltiplos testes, tanto em ambiente simulado usando
    o utilitário da linha de comandos \textit{socat} e o ficheiro de exemplo fornecido,
    \textit{cable.c}, como em ambiente laboratorial no laboratório I321 na
    \textit{FEUP}. Em ambiente laboratorial foi testado o envio de diversos ficheiro
    de 4 modos distintos:

    \begin{itemize}
        \item Sem interrupções e sem interferências.
        \item Com interrupções mas sem interferências.
        \item Sem interrupções mas com interferências.
        \item Com interrupções e com interferências.
    \end{itemize}

    Em todas as situações, os protocolados provaram ser robustos uma vez que garantiram o
    correto envio sem erros dos ficheiros enviados. Esta certeza foi garantida através do
    uso do utilitário da linha de comandos \textit{diff} que foi utilizado para comparar
    na máquina recetora o ficheiro reebido com uma cópia que a mesma já tinha do ficheiro. \par

    Foi também testada a ordem de execução dos programas: \textit{receiver} seguido pelo
    \textit{sender} e \textit{sender} seguido pelo \textit{receiver} obtendo
    em ambas as situações um correto funcionamento do envio de ficheiros.

%%% Eficiência do protocolo de ligação de dados %%%
\section{Eficiência do protocolo de ligação de dados}

%TODO


%%% Conclusões %%%
\section{Conclusões}

    Foram implementados em C dois protocolos robustos, ligação de dados e aplicação, para
    transferir ficheiros entre computadores usando a porta série. Foi garantida
    independência entre camadas e a correta implementação do mecanismo de
    \textit{Stop-and-Wait} para controlo de erros no protocolo de ligação
    de dados. Foi ainda escrito este relatório que inclui uma análise de eficiência.
    Assim, foram cumpridos todos os objetivos deste projeto. \par

    A realização deste projeto permitiu lidar na prática com os detalhes abordados
    teóricamente nas aulas que de outra forma nos passariam despercebidos. Assim sendo,
    a sua conceção demonstrou ser uma forma de estudo imersiva dos conteúdos lecionados
    em \textit{Redes de Computadores}.


%%% Anexos %%%
\newpage

\section{Anexos}
\subsection{Código Fonte}

\lstinputlisting[language=C, caption=sender.c]{../src/sender.c}

\lstinputlisting[language=C, caption=receiver.c]{../src/receiver.c}

\lstinputlisting[language=C, caption=aplic.h]{../src/aplic.h}

\lstinputlisting[language=C, caption=linklayer.h]{../src/linklayer.h}

\lstinputlisting[language=C, caption=aplic.c]{../src/aplic.c}

\lstinputlisting[language=C, caption=linklayer.c]{../src/linklayer.c}


\end{document}
