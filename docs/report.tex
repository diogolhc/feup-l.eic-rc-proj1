\documentclass[a4paper,11pt,portuguese]{article}
\usepackage{hyperref}
\usepackage{listings}
\usepackage{xcolor}

%%% Estilo de código %%%

\definecolor{codegreen}{rgb}{0,0.6,0}
\definecolor{codegray}{rgb}{0.5,0.5,0.5}
\definecolor{codepurple}{rgb}{0.58,0,0.82}
\definecolor{backcolour}{rgb}{0.95,0.95,0.92}

\lstdefinestyle{mystyle}{
    backgroundcolor=\color{backcolour},   
    commentstyle=\color{codegreen},
    keywordstyle=\color{magenta},
    numberstyle=\tiny\color{codegray},
    stringstyle=\color{codepurple},
    basicstyle=\ttfamily\footnotesize,
    breakatwhitespace=false,         
    breaklines=true,                 
    captionpos=b,                    
    keepspaces=true,                 
    numbers=left,                    
    numbersep=5pt,                  
    showspaces=false,                
    showstringspaces=false,
    showtabs=false,                  
    tabsize=2
}

\lstset{style=mystyle}


\begin{document}

%%% Identificação %%%

\author{
    Diogo Costa\\
    \href{mailto:up201906731@edu.fe.up.pt}{up201906731@edu.fe.up.pt}
    \and
    Francisco Colino\\
    \href{mailto:up201905405@edu.fe.up.pt}{up201905405@edu.fe.up.pt}
}
\title{Redes de Computadores \\ 1.º Trabalho Prático}
\date{11 de dezembro 2021}
\maketitle

%%% Sumário %%%
\begin{center}
    \textbf{Sumário}
\end{center}

%TODO


%%% Introdução %%%
\section{Introdução}

%TODO


%%% Arquitetura %%%
\section{Arquitetura}

%TODO


%%% Estrutura do código %%%
\section{Estrutura do código}

%TODO


%%% Casos de uso principais %%%
\section{Casos de uso principais}

%TODO


%%% Protocolo de ligação lógica %%%
\section{Protocolo de ligação lógica}

%TODO


%%% Protocolo de aplicação %%%
\section{Protocolo de aplicação}

%TODO


%%% Validação %%%
\section{Validação}

%TODO


%%% Eficiência do protocolo de ligação de dados %%%
\section{Eficiência do protocolo de ligação de dados}

%TODO


%%% Conclusões %%%
\section{Conclusões}

%TODO


%%% Anexos %%%
\newpage

\section{Anexos}
\subsection{Código Fonte}

\lstinputlisting[language=C, caption=sender.c]{../src/sender.c}

\lstinputlisting[language=C, caption=receiver.c]{../src/receiver.c}

\lstinputlisting[language=C, caption=aplic.h]{../src/aplic.h}

\lstinputlisting[language=C, caption=linklayer.h]{../src/linklayer.h}

\lstinputlisting[language=C, caption=aplic.c]{../src/aplic.c}

\lstinputlisting[language=C, caption=linklayer.c]{../src/linklayer.c}


\end{document}
